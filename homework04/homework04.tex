\documentclass{homework}
\usepackage{cancel}
\usepackage{amsthm}
\usepackage{cleveref}
\usepackage{upgreek}
\usepackage[framed]{mcode}
\usepackage{mathrsfs}
\usepackage{tikz}
\usepackage{units}
\usetikzlibrary{matrix}
\newtheorem{lemma}{Lemma}
\DeclareMathOperator*{\argmin}{arg\,min}

\title{Kevin Joyce}
\course{Math 514 - Inverse Problems - Homework 4}
\author{Kevin Joyce}
\docdate{\today}
\begin{document} 
\newcommand{\figref}[1]{\figurename~\ref{#1}}
\renewcommand{\bar}{\overline}
\renewcommand{\hat}{\widehat}
\renewcommand{\SS}{\mathcal S}
\newcommand{\HH}{\mathscr H}
\newcommand{\mom}{\widetilde}
\newcommand{\mle}{\widehat \Uptheta}
\newcommand{\eps}{\varepsilon}
\newcommand{\todist}{\stackrel{D}\longrightarrow}
\newcommand{\toprob}{\stackrel{p}\longrightarrow}
\newcommand{\TTheta}{\overline{\underline \Theta} }
\newcommand{\del}{\partial}
\newcommand{\approxsim}{\overset{\cdotp}{\underset{\cdotp}{\sim}}}

\begin{longproblem} 
Bardsley 2.9. In exercise 1.2, you were asked to modify \texttt{Deblur1d.m} so that the convolution kernel
$$
  a(s) = \begin{cases}
  100s + 10,  &-\frac 1{10} \le s < 0,\\
  -100s + 10, &0\le s \le \frac 1{10},\\
  0,	      &\text{otherwise}
  \end{cases}
$$
is used instead to define $\vect A$.

\subproblem{ Use Tikhonov regularization together with GCV and L-curve to reconstruct $\vect x$ from observations $\vect b$.  What is the optimal regularization parameter $\alpha$ in each case?  Which gives the better reconstruction in your opinion?}

\subproblem{Use TSVD regularization together with UPRE and DP to reconstruct $\vect x$ from observations $\vect b$.  What is the optimal regularization parameter $k$ in each case?  Which gives the better reconstruction in your opinion?}
\end{longproblem}

\problem{Bardsley 3.1. Modify \texttt{OnedDeblurBCs.m} so that it implements GCV and UPRE regularization parameter selection methods.  How do these parameter selection methods perform in terms of the visual quality of the regularized reconstructions?}
\begin{solution}
\end{solution}

\problem{Bardsley 3.5. Use the Kronecker product properties (3.12)-(3.14) to prove (3.15) and (3.16)}

\problem{Bardsley 3.6a. Derive the formulas for GCV analogous to (3.18).  Add lines of code to \texttt{Deblur2dSeparable.m} so that it implements GCV.}
\end{document} 
